% Capítulo 1 - Introdução
% Exemplo de estrutura de capítulo

\startchapter[title={Introdução},reference=chap:intro]

\startsection[title={Visão Geral}]

Este é um exemplo de documento criado com o framework \color[CAPPrimary]{\bf ConTeXt Academic Press}.
O framework foi desenvolvido para facilitar a criação de livros didáticos e materiais acadêmicos
com design elegante e profissional.

\startsubsection[title={Recursos Principais}]

Os principais recursos incluem:

\startitemize[packed]
  \item Design moderno e minimalista
  \item Tipografia profissional com fontes Google
  \item Sistema de cores configurável
  \item Ambientes especiais para teoremas, exemplos e exercícios
  \item Suporte completo para matemática de alta qualidade
  \item Exportação para múltiplos formatos
\stopitemize

\stopsubsection

\stopsection

\startsection[title={Exemplos de Elementos}]

\startsubsection[title={Teoremas e Definições}]

Aqui está um exemplo de teorema:

\starttheorem[title={Teorema de Pitágoras}]
Em um triângulo retângulo, o quadrado da hipotenusa é igual à soma dos quadrados dos catetos.

\startformula
  c^2 = a^2 + b^2
\stopformula
\stoptheorem

E uma definição:

\startdefinition[title={Função Contínua}]
Uma função \math{f: \reals \to \reals} é contínua em \math{x_0} se para todo \math{\epsilon > 0},
existe \math{\delta > 0} tal que:

\startformula
  |x - x_0| < \delta \Rightarrow |f(x) - f(x_0)| < \epsilon
\stopformula
\stopdefinition

\stopsubsection

\startsubsection[title={Exemplos e Exercícios}]

\startexample[title={Cálculo de Limite}]
Calcule o limite:
\startformula
  \lim_{x \to 2} \frac{x^2 - 4}{x - 2}
\stopformula

\startsolution
Fatorando o numerador:
\startformula
  \frac{x^2 - 4}{x - 2} = \frac{(x-2)(x+2)}{x-2} = x + 2
\stopformula

Portanto:
\startformula
  \lim_{x \to 2} \frac{x^2 - 4}{x - 2} = \lim_{x \to 2} (x+2) = 4
\stopformula
\stopsolution
\stopexample

\startexercise
Calcule os seguintes limites:
\startitemize[n]
  \item \math{\displaystyle\lim_{x \to 3} \frac{x^2 - 9}{x - 3}}
  \item \math{\displaystyle\lim_{x \to 0} \frac{\sin x}{x}}
  \item \math{\displaystyle\lim_{x \to \infty} \frac{2x^2 + 1}{x^2 - 3}}
\stopitemize
\stopexercise

\stopsubsection

\startsubsection[title={Boxes Especiais}]

Você pode usar diferentes tipos de boxes para destacar conteúdo:

\startCAPNote
Esta é uma nota informativa. Use para informações importantes que complementam o texto principal.
\stopCAPNote

\startCAPWarning
Este é um aviso. Use para alertar sobre conceitos que frequentemente causam confusão ou
erros comuns.
\stopCAPWarning

\stopsubsection

\stopsection

\startsection[title={Matemática Avançada}]

O framework suporta matemática complexa com excelente qualidade tipográfica.

\startsubsection[title={Equações}]

Equações numeradas automaticamente:

\startformula
  \int_{a}^{b} f(x) \, dx = F(b) - F(a)
\stopformula

\startformula
  \sum_{n=1}^{\infty} \frac{1}{n^2} = \frac{\pi^2}{6}
\stopformula

\stopsubsection

\startsubsection[title={Matrizes}]

Exemplo de matriz:

\startformula
  A = \startpmatrix
    \NC 1 \NC 2 \NC 3 \NR
    \NC 4 \NC 5 \NC 6 \NR
    \NC 7 \NC 8 \NC 9 \NR
  \stoppmatrix
\stopformula

\stopsubsection

\stopsection

\startsection[title={Próximos Passos}]

Para aprender mais sobre o framework, consulte a documentação completa em \type{docs/}.

Alguns tópicos importantes:
\startitemize
  \item Configuração de cores e fontes
  \item Criação de templates personalizados
  \item Módulos especializados por área
  \item Compilação e exportação
\stopitemize

\stopsection

\stopchapter
